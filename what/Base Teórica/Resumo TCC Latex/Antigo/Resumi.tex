\documentclass[a4paper,12p]{book}
\setcounter{tocdepth}{3}
\setcounter{secnumdepth}{3}
\usepackage[a4paper, inner=0.7cm, outer=1.7cm, top=2cm, bottom=2cm, bindingoffset=1.2cm]{geometry}
\usepackage[portuguese]{babel}
\usepackage{blindtext}
\usepackage{microtype}
\usepackage{graphicx}
\usepackage{wrapfig}
\usepackage{enumitem}
\usepackage{fancyhdr}
\usepackage{amsmath}
\usepackage{index}
\usepackage{multicol,float,listings}
\usepackage{setspace}
\usepackage[super]{nth}
\makeindex

\begin{document}
	
	%\title{\LARGE{\textbf{Historia Immeral Varis}}}
	%\author{Luan Henrique Sirtoli}
	%\date{27 de Março de 2019}
	%\maketitle
	
	%\tableofcontents
	
	%\pagenumbering{roman}
	\setcounter{page}{1}
	
	
	\chapter{Immeral Varis}
	\section{Background}
	
	Immeral vem de uma família de Mercantes Nômades, que vive a vida comprando itens pelas cidades do mundo e revendendo por melhores valores. Por trabalharem com altos valores, foram marcados por ladrões e assassinos, e durante uma viagem por uma floresta foram atacados e mortos enquanto ele era relativamente novo. Assim como instruído por seus pais, fugiu e se embrenhou na floresta, sendo salvo pelos animais e espíritos da floresta. Porém, por algum motivo magicamente desconhecido, ele esteve preso nessa floresta até se tornar um adulto. Isso lhe deu tempo para se conectar com os espíritos e os animais. Infelizmente o tempo longe da civilização o tornou mais ingênuo e mais selvagem. Assim, para se proteger e proteger as florestas, os espíritos lhe ensinaram a arte Druídica.
	
	Por estar preso dentro de uma floresta; por não ter pais e amigos por perto, para não enlouquecer, Immeral teve que se apegar a alguma coisa, assim adotou um filhote de animal e tomou conta dele todo esse tempo.
	
	Assim, adotou um pequeno filhote de pantera, deu o nome de "Bast" para ela, e essa se tornou uma grande companheira.
	
	\section{Aparência}
	\begin{multicols}{2}
		
	Immeral é um Elfo da Floresta, alto, com a seguinte aparência:
	
	\textbf{Altura:} Aproximadamente 1.92m.
	
	\textbf{Peso:} 92Kg.
	
	\textbf{Cabelos:} Loiros e raspados na lateral.
	
	\begin{figure}[H]
		\centering
		\includegraphics[width=7cm]{E:/Downloads/770a21f153c3cc9cb9eee10f0e6f8fd6.jpg}
		Aparência de Immeral Varis
	\end{figure}
	
	
	%\vfill\null
	%\columnbreak
	
	Bast é uma Pantera Negra, Forte, com a seguinte aparência:
	
	\textbf{Altura:} Deitada: 0.90m, Pé: 1.50m.
	
	\textbf{Peso:} 75Kg.
	
	\textbf{Pelugem:} Negra.
	
	\begin{figure}[H]
		\centering
		\includegraphics[width=7cm]{E:/Downloads/e54a8849f5940a24b4bd5768e9ee4989.jpg}
		Aparência de Bast, sua Pantera
	\end{figure}

	\end{multicols}
	
	
	
	
	
	
	
	
	
	
	
	\section{Fichas Iniciais}
	
	\begin{multicols}{2}
		\subsection{Immeral Varis}
		\textit{Elfo Florestal, Caótico e Neutro, Eremita}

		\begin{itemize}
			\item \textbf{Classe de Armadura:} 14
			\item \textbf{Pontos de Vida:} 10
			\item \textbf{Deslocamento:} 10.5m
			\item \textbf{Escalada:} 10.5m
		\end{itemize}
		\begin{center}
			\resizebox{\columnwidth}{!}{%
				\begin{tabular}{|l|l|l|l|l|l| p{5cm} }
					\hline
					\textbf{FOR} & \textbf{DES} & \textbf{CON} & \textbf{INT} & \textbf{SAB} & \textbf{CAR} \\ \hline
					08(-1)      & 16(+3)      & 14(+2)      & 10(+0)       & 16(+3)      & 10(+0)       \\ \hline
			\end{tabular}}
		\end{center}

		\paragraph{Características}
		
		\begin{itemize}
			
			\item \textbf{Perícias:}
			
			\begin{itemize}
				\item Natureza +2;
				\item Lidar com Animais +5;
				\item Medicina +5;
				\item Percepção +5;
			\end{itemize}
		
			\item \textbf{Sentidos:} Percepção passiva 15
			
			\item \textbf{Idiomas:} Comum, Druídico e Elfo e mais um.
			
			\item \textbf{Nível de Desafio:} -
		
			%\item \textbf{Bote:} Se a pantera se mover, pelo menos, 6 metros em linha reta em direção de um alvo logo antes de atingi-lo com seu ataque de garra, o alvo deve ser bem sucedido num \textbf{teste de resistência de Força CD 12} para não cair no chão. Se o alvo cair no chão, a pantera poderá realizar uma ação bônus para realizar um ataque de mordida contra ele. 
		
			%\item \textbf{Faro Aguçado:} A pantera tem vantagem em \textbf{testes de Sabedoria (Percepção)} relacionados ao olfato.
		
		
		\end{itemize}
	
		\paragraph{Ações}

		
		\begin{itemize}
			\item \textbf{Ataque com Cimitarra:} 
			\begin{itemize}
				\item \textbf{Propriedades:} Acuidade, Leve;
				\item \textbf{Ataque Corpo-a-Corpo com Arma:} +4 para atingir;
				\item \textbf{Alcance:} 1,5 m, um alvo; 
				\item \textbf{Acerto:} (1d6 + For ou Des) de dano cortante.
			\end{itemize}
		
			\item \textbf{Ataque com Bordão:}
			\begin{itemize}
				\item \textbf{Propriedades:} Versátil (1d8);
				\item \textbf{Ataque Corpo-a-Corpo com Arma:} +4 para atingir;
				\item \textbf{Alcance:} 1,5 m, um alvo; 
				\item \textbf{Acerto 1 mão:} (1d6 + For) de dano por concussão.
				\item \textbf{Acerto 2 mãos:} (1d8 + For) de dano por concussão.
				\item \textbf{Magia:} Utilizando 'Bordão Místico' pode-se trocar o dano de For por Sab.
			\end{itemize}
		\end{itemize}
		
		
		
		
		
		
		\vfill\null
		%\vfill\null %PARA MULTICOLUMN
		%\vfill\eject %PARA TWOCOLUMN
		\columnbreak


		\subsection{Bast}
		\textit{Besta Média, imparcial}
		
		\begin{itemize}
			\item \textbf{Classe de Armadura:} 12
			\item \textbf{Pontos de Vida:} 13
			\item \textbf{Deslocamento:} 15m
			\item \textbf{Escalada:} 12m
		\end{itemize}
		\begin{center}
			\resizebox{\columnwidth}{!}{%
				\begin{tabular}{|l|l|l|l|l|l| p{5cm} }
					\hline
					\textbf{FOR} & \textbf{DES} & \textbf{CON} & \textbf{INT} & \textbf{SAB} & \textbf{CAR} \\ \hline
					14(+2)      & 15(+2)      & 10(+0)      & 03(-4)       & 14(+2)      & 07(-2)       \\ \hline
			\end{tabular}}
		\end{center}
		
		\paragraph{Características}
		
		\begin{itemize}
			
			\item \textbf{Perícias:}
			
			\begin{itemize}
				\item Furtividade +6; 
				\item Percepção +4;
			\end{itemize}
			
			\item \textbf{Sentidos:} Percepção passiva 14
			
			\item \textbf{Idiomas:} –
			
			\item \textbf{Nível de Desafio:} 1/4 (50 XP)
			
			\item \textbf{Bote:} Se a pantera se mover, pelo menos, 6 metros em linha reta em direção de um alvo logo antes de atingi-lo com seu ataque de garra, o alvo deve ser bem sucedido num \textbf{teste de resistência de Força CD 12} para não cair no chão. Se o alvo cair no chão, a pantera poderá realizar uma ação bônus para realizar um ataque
			de mordida contra ele. 
			
			\item \textbf{Faro Aguçado:} A pantera tem vantagem em \textbf{testes de Sabedoria (Percepção)} relacionados ao olfato.
			
			
		\end{itemize}
		
		\paragraph{Ações}
		
		
		\begin{itemize}
			\item \textbf{Mordida:} 
			\begin{itemize}
				\item \textbf{Ataque Corpo-a-Corpo com Arma:} +4 para atingir;
				\item \textbf{Alcance:} 1,5 m, um alvo; 
				\item \textbf{Acerto:} 5 (1d6 + 2) de dano perfurante.
			\end{itemize}
			
			\item \textbf{Garras:} 
			\begin{itemize}
				\item \textbf{Ataque Corpo-a-Corpo com Arma:} +4 para atingir;
				\item \textbf{Alcance:} 1,5 m, um alvo; 
				\item \textbf{Acerto:} 4 (1d4 + 2) de dano cortante.
			\end{itemize}
		\end{itemize}
	
	
			
	\vfill\null
	\end{multicols}

	\pagebreak








	\section{Vantagens}
	\begin{multicols}{2}
	\subsection{Vantagens Élficas}
		\begin{itemize}
			\item \textbf{Aumento no Valor de Habilidade.} Seu valor de
			Destreza aumenta em 2.
			\item \textbf{Deslocamento.} Seu deslocamento base de caminhada
			é 9 metros.
			\item \textbf{Visão no Escuro}. Acostumado às florestas
			crepusculares e ao céu noturno, você possui uma visão
			superior em condições de escuridão e na penumbra. Você
			pode enxergar na penumbra a até 18 metros como se fosse
			na luz plena, e no escuro como se fosse na penumbra.
			Você não pode discernir cores no escuro, apenas tons de
			cinza.
			\item \textbf{Sentidos Aguçados. }Você tem proficiência na perícia
			Percepção.
			\item \textbf{Ancestral Feérico}. Você tem vantagem nos testes de
			resistência para resistir a ser enfeitiçado e magias não
			podem colocá-lo para dormir.
			\item \textbf{Transe.} Elfos não precisam dormir. Ao invés disso,
			eles meditam profundamente, permanecendo
			semiconscientes, durante 4 horas por dia. (A palavra em
			idioma comum para tal meditação é "transe".) Enquanto
			medita, um elfo é capaz de sonhar de certo modo. Esses
			sonhos na verdade são exercícios mentais que se tornam
			reflexos através de anos de prática. Depois de descansar
			dessa forma, você ganha os mesmos benefícios que um
			humano depois de 8 horas de sono.
			\item \textbf{Idiomas.} Você pode falar, ler e escrever Comum e
			Élfico. O Élfico é um idioma fluido, com entonações sutis e
			gramática complexa. A literatura élfica é rica e diversa, e
			suas canções e poemas são famosos entre outras raças.
			Muitos bardos aprendem essa língua para que possam
			adicionar canções élficas ao seu repertório.
		\end{itemize}
	
		\subsection{Vantagens Élficas - Floresta}
			\begin{itemize}
				\item \textbf{Aumento no Valor de Habilidade.} Seu valor de
				Sabedoria aumenta em 1.
				Treinamento Élfico com Armas. Você possui
				proficiência com espadas longas, espadas curtas, arcos
				longos e arcos curtos.
				\item \textbf{Pés Ligeiros.} Seu deslocamento base de caminhada
				aumenta para 10,5 metros.
				\item \textbf{Máscara da Natureza.} Você pode tentar se esconder
				mesmo quando você está apenas levemente obscurecido
				por folhagem, chuva forte, neve caindo, névoa ou outro
				fenômeno natural.
			\end{itemize}
	
	\columnbreak
	
	%{\setstretch{0.95}
	\subsection{Vantagens Druídicas}
		
		\begin{itemize}
			\item \textbf{Dado de Vida:} 1d8 por nível de druida
			\item \textbf{Druidico:} Você conhece o Druídico, o idioma secreto dos druidas. Você pode falar esse idioma e usá-lo para deixar mensagens escondidas. Você e outros que conheçam esse idioma automaticamente veem tais mensagens. Outros	perceberão a presença da mensagem se passarem num teste de Sabedoria (Percepção) CD 15, mas não conseguirão decifrá-lo sem magia.
		\end{itemize}
	\subsubsection{Proficiências:}
		\begin{itemize}
			\item \textbf{Armaduras:} Armaduras leves, armaduras médias, escudos (druidas não irão vestir armaduras ou usar escudos feitos de metal)
			\item \textbf{Armas:} Clavas, adagas, dardos, azagaias, maças, bordões, cimitarras, foices, fundas e lanças.
			\item \textbf{Ferramentas:} Kit de herbalismo.
			\item \textbf{Testes de Resistência:} Inteligência, Sabedoria;
			\item \textbf{Perícias:} Escolha duas dentre Arcanismo, Adestrar Animais, Intuição, Medicina, Natureza, Percepção, Religião e Sobrevivência;
		\end{itemize}
	\subsubsection{Preparando e Conjurando Magias:}
	Você prepara a lista de magias disponíveis selecionando-as da lista de magias de Druida. Você seleciona um número de magias igual ao seu modificador de Sabedoria + seu nível de druida (mínimo de uma magia). Essas magias devem ser de níveis que você possua espaços de magia.
	
	Sabedoria é a sua habilidade para você conjurar suas magias de druida, já que sua magia vem da sua devoção e sintonia com a natureza. Você usa sua Sabedoria sempre que alguma magia se referir a sua habilidade de conjurar magias. Além disso, você usa o seu modificador de Sabedoria para definir a CD dos testes de resistência para as magias de druida que você conjura e quando você realiza uma jogada de ataque com uma magia.
	
	\begin{itemize}
		\item \textbf{CD para suas magias} = 8 + bônus de proficiência +
		seu modificador de Sabedoria
		
		\item \textbf{Modificador de ataque de magia} = seu bônus de proficiência + seu modificador de Sabedoria
	\end{itemize}

	\columnbreak%}
	\subsubsection{Forma Selvagem:}
	
	A partir do \textbf{2\textsuperscript{o} nível}, você pode usar sua ação para assumir magicamente a forma de uma besta que você já tenha visto antes. Você pode usar essa característica duas vezes. Você recupera os usos quando termina um descanso curto ou longo.
	
	
	Seu nível de druida determina as bestas em que você pode se transformar, como mostrado na tabela Formas de Besta. No \textbf{2\textsuperscript{o} nível}, por exemplo, você pode se transformar em qualquer besta que possui nível de desafio 1/4 ou inferior que não possua deslocamento de voo ou natação.
	
	
	\begin{center}
		\centering
		\resizebox{\columnwidth}{!}{%
		\begin{tabular}{|l|l|l|}
			\hline
			\textbf{Nível} & \textbf{ND Máx.} & \textbf{Limitações}                \\ \hline
			2\textsuperscript{o}             & 1/4              & Sem deslocamento de voo ou natação \\ \hline
			4\textsuperscript{o}             & 1/2              & Sem deslocamento de voo            \\ \hline
			8\textsuperscript{o}             & 1                & -                                  \\ \hline
		\end{tabular}}
	\end{center}

	Você pode continuar na forma de besta por um número de horas igual à metade do seu nível de druida (arredondado para baixo). Então, você volta a sua forma original, a não ser que você gaste outro uso dessa característica. Você pode reverter a sua forma normal prematuramente usando uma ação bônus no seu turno. Você reverte automaticamente se cair inconsciente, cair a 0 pontos de vida ou morrer.
	
	Enquanto estiver transformado, as seguintes regras se aplicam:
	
	\begin{itemize}
		\item Suas estatísticas de jogo são substituídas pelas estatísticas da besta, mas você mantem sua tendência, personalidade e valores de Inteligência, Sabedoria e Carisma. Você também mantem suas proficiências em todas as suas perícias e testes de resistência, além de receber as proficiências da criatura. Se a criatura possuir a mesma proficiência que você e o bônus no bloco de estatística dela for maior que o seu, você usará o bônus da criatura no lugar do seu. Se a criatura possuir qualquer ação lendária ou de covil, você não pode usá-las.
		
		\item Quando você se transforma, você assume os pontos de vida e Dados de Vida da criatura. Quando você reverte a sua forma normal, você retorna ao número de pontos de vida que tinha antes de se transformar. Porém, se você reverter como resultado de ter caído a 0 pontos de vida, todo o dano excedente será transferido para a sua forma normal. Por exemplo, se você sofrer 10 pontos de dano em forma animal e tiver apenas 1 ponto de vida restante, você reverte e sofre 9 de dano. Contanto que o dano excedente não reduza você a 0 pontos de vida, você não cairá inconsciente.
		
		\item Você não pode conjurar magias e sua capacidade de fala ou de realizar qualquer ação que requeira mãos são limitadas pelas capacidades da forma da besta que você assumiu. Transformar-se não interrompe sua concentração em uma magia que você já tenha conjurado, no entanto, nem previne você de realizar ações que são parte da conjuração, como convocar relâmpagos que você já tenha conjurado.
		
		
		\columnbreak
		
		\item Você mantem os benefícios de todas as características de classe, raça ou outras fontes, e pode usá-las caso a nova forma seja fisicamente capaz de fazê-lo. No entanto, você não pode usar qualquer dos seus sentidos especiais, como visão no escuro, a não ser que a sua nova forma também tenha esse sentido.
		
		%\vfill\null
		\item Você pode escolher se o seu equipamento cai no chão no seu espaço, é assimilado a sua nova forma ou é usado por ela. Equipamentos vestidos e carregados funcionam normalmente, mas o Mestre decide qual equipamento é viável para a nova forma vestir ou usar, baseado na forma e tamanho da criatura. O seu equipamento não muda de forma ou tamanho para se adaptar à nova forma e, qualquer equipamento que a nova forma não possa vestir deve, ou cair no chão ou ser assimilado por ela. Equipamentos assimilados não terão efeito até você deixar a forma.
	\end{itemize}

	\subsubsection{Habilidades Por Nível}
	\begin{itemize}
		\item \textbf{Círculo Druídico} 
		
		No \textbf{2\textsuperscript{o} nível}, você escolhe se identificar com um círculo de druidas: o Círculo da Terra ou o Círculo da Lua, ambos detalhados no final da descrição da classe. Sua escolha lhe concede características no \textbf{2\textsuperscript{o} nível} e novamente no \textbf{6\textsuperscript{o}, 10\textsuperscript{o} e 14\textsuperscript{o} nível}.
		
		\item \textbf{Incremento no Valor de Habilidade}
		
		Quando você atinge o \textbf{4\textsuperscript{o} nível} e novamente no \textbf{8\textsuperscript{o}, 12\textsuperscript{o}, 16\textsuperscript{o} e 19\textsuperscript{o} nível,} você pode aumentar um valor de habilidade, à sua escolha, em 2 ou você pode aumentar dois valores de habilidade, à sua escolha, em 1. Como padrão, você não
		
		\item \textbf{Corpo Atemporal}
		
		Começando no \textbf{18\textsuperscript{o} nível}, a magia primordial que você controla faz com que você envelheça mais lentamente. Para cada 10 anos que passarem, seu corpo envelhece apenas 1.
		
		\item \textbf{Magias da Besta}
		
		A partir do \textbf{18\textsuperscript{o} nível}, você pode conjurar muitas das suas magias em qualquer forma que assumir usando a Forma Selvagem. Você pode realizar os componentes somáticos e verbais de uma magia de druida na forma de besta, mas você não é capaz de prover os componentes materiais.
		
		\item \textbf{Arquidruida}
		
		No \textbf{20\textsuperscript{o} nível}, você pode usar sua Forma Selvagem um número ilimitado de vezes. Além disso, você pode ignorar os componentes verbais e somáticos das suas magias de druida, assim como qualquer componente material que não tenha custo e não seja consumido pela magia. Você recebe esse benefício tanto na sua forma normal, quanto na forma de besta da sua Forma Selvagem.
	\end{itemize}
	
	\columnbreak
	\subsection{Vantagens Ancestrais}
	\begin{itemize}
		\item \textbf{Proficiência em Perícias:} Medicina, Religião
		
		\item \textbf{Proficiência em Ferramentas:} 
		Kit de herbalismo. \textbf{\textit{(Trocado por Kit de Venenos).}}
		
		\item \textbf{Idiomas:} Um à sua escolha
		
		\item \textbf{Equipamento:} Um estojo de pergaminho cheio de notas dos seus estudos e orações, um cobertor de inverno, um conjunto de roupas comuns, um kit de herbalismo e 5 po.
		
		\item \textbf{Característica - Descoberta:}
		
		A calma reclusão da seu eremitério prolongado lhe deu acesso a descobertas únicas e poderosas. A natureza exata dessas revelações depende da natureza da sua reclusão. Poderia ser uma grande verdade sobre o cosmos, os deuses, os poderosos seres de outros planos ou as forças da natureza. Poderia ser um local nunca visto por mais ninguém. Você pode ter descoberto um fato que a muito foi esquecido, ou desenterrado uma relíquia do passado que poderia reescrever a história. Poderia ser uma informação que seria prejudicial para as pessoas responsáveis pelo seu exilio, consequentemente, a razão que fez você voltar para a sociedade.
		
		
	\end{itemize}
	
	\subsubsection{Perfil Ancestral}
	
	\begin{itemize}
		\item \textbf{Vida de Isolamento:} Eu me afastei da sociedade após um evento que mudou minha vida.
		
		\item \textbf{Traço de Personalidade:} Eu relaciono tudo que acontece comigo a um grande plano cósmico.
		
		\item \textbf{Ideal:} Pensamento Livre - Questionamentos e curiosidade são os pilares do progresso.
		
		\item \textbf{Vínculo:} Meu isolamento me deu grande discernimento sobre um grande mal que apenas eu posso destruir.
		
		\item \textbf{Defeito:} Eu gosto de guardar segredos e não os partilho com ninguém.
		
		
		
	\end{itemize}
	%\vfill\null
	
	
	
	\end{multicols}
	\begin{figure}[H]
		\centering
		\includegraphics[width=\columnwidth]{E:/Downloads/Druida.png}
	\end{figure}
	%\pagebreak
	
	
	\section{Círculos Druídicos}
		Apesar de suas organizações serem invisíveis para a maioria dos forasteiros, os druidas fazem parte de uma sociedade que se espalham pela terra, ignorando barreiras políticas. Todos os druidas são nominalmente membros de uma sociedade druídica, apesar de alguns indivíduos serem tão isolados que eles nunca chegaram a ver membros de alta patente da sociedade ou participaram de encontros druídicos. Os druidas consideram-se irmãos e irmãs. Como criaturas na natureza, no entanto, os druidas, as vezes, competem, ou mesmo caçam uns aos outros.
		Em uma escala local, os druidas são organizados em círculos que partilham de certas perspectivas de natureza, equilíbrio e modos de um druida.
		
		{\setstretch{0.9}
		\subsection{Círculo da Terra}
		O Círculo da Terra é constituído por místicos e sábios que salvaguardam conhecimento e ritos antigos através de uma vasta tradição oral. Esses druida se encontram em círculos sagrados de árvores ou monólitos para sussurrar segredos primordiais em Druídico. Os membros mais sábios do círculo presidem como os sacerdotes-dirigentes de comunidades que creem na Crença Antiga, e servem como conselheiros para os governantes desses povos. Como membro desse círculo, sua magia é influenciada pela terra onde você é iniciado nos ritos misteriosos do círculo.
		
		\begin{multicols}{2}
				\begin{itemize}
				\item \textbf{Truque Adicional.} Quando você escolhe esse círculo no Segundo nível, você aprende um truque de druida adicional, à sua escolha.
				
				\item \textbf{Recuperação Natural.} A partir do \textbf{2\textsuperscript{o} nível}, você pode recuperar parte da sua energia mágica parando para fazer uma meditação e comunhão com a natureza. Durante um descanso curto, você escolhe espaços de magia gastos para recuperar. O espaço de magia pode ter um nível combinado igual ou menor que metade do seu nível de druida (arredondado para baixo) e, nenhum dos espaços pode ser de uma magia de \textbf{6\textsuperscript{o} nível} ou superior. Você não pode usar essa característica novamente até terminar um descanso longo.
				Por exemplo, quando você for um druida de \textbf{4\textsuperscript{o} nível}, você pode recuperar até dois níveis em espaços de magia. Você pode recuperar, tanto uma magia de \textbf{2\textsuperscript{o} nível}, quanto duas magias de \textbf{1\textsuperscript{o} nível}.
				
				\item \textbf{Magias De Círculo.} Sua conexão mística com a terra infunde você com a habilidade de conjurar certas magias. No \textbf{3\textsuperscript{o}, 5\textsuperscript{o}, 7\textsuperscript{o} e 9\textsuperscript{o} nível}, você ganha acesso a magias de círculo ligadas ao terreno em que você se tornou druida. Escolha o terreno – ártico, costa, deserto, floresta, montanha, pântano, planície ou subterrâneo – e consulte a lista de magias associada.
				
				Uma vez que você tenha acesso a uma magia de círculo, você sempre poderá prepará-la e ela não conta no número de magias que você pode preparar a cada dia. Se você tiver acesso a uma magia que não aparece na lista de magias de druida, a magia, no entanto, será uma magia de druida para você.

				\item \textbf{Caminho da Floresta.} A partir do \textbf{6\textsuperscript{o} nível}, mover-se através de terreno difícil não-mágico não te custará nenhum movimento extra. Você também pode passar através de plantas não-mágicas sem ser atrasado por elas e sem sofrer dano delas se elas tiverem espinhos, espinhas ou perigos similares.
				Além disso, você tem vantagem em testes de resistência contra plantas criadas magicamente ou manipuladas para impedir movimentação, como as criadas pela magia constrição.
				
				\item \textbf{Proteção Natural.} Quando você atingir o \textbf{10\textsuperscript{o} nível}, você não pode ser enfeitiçado ou amedrontado por elementais ou fadas e você se torna imune a venenos e doenças.
				
				\item \textbf{Santuário Natural.} A partir do \textbf{14\textsuperscript{o} nível}, as criaturas do mundo natural sentem sua ligação com a natureza e hesitarão em atacar você. Quando uma besta ou plantar atacar você, essa criatura deverá fazer um teste de resistência de Sabedoria contra uma CD igual a das suas magias de druida. Em uma falha, a criatura deve escolher um alvo diferente ou o ataque erra automaticamente. Em um sucesso, a criatura se torna imune a esse efeito por 24 horas.
				A criatura está ciente deste efeito antes de resolver atacar você.
				
				\end{itemize}
			%\columnbreak
			\end{multicols}
			%\pagebreak
		
			
			\subsection{Círculo da Lua}
			Os druidas do Círculo da Lua são ferrenhos guardiões na natureza. Sua ordem se reuni nas noites de lua cheia para partilhar notícias e trocar informações. Eles assombram as partes mais profundas das florestas, onde eles podia ir por semanas a fio antes de cruzar o caminho de outro humanoide e, muito menos outro druida.
			Tão mutável quanto a lua, um druida desse círculo poderia espreitar como um grande felino, voar sobre a copa das árvores como uma águia no dia seguinte e mergulhar pela vegetação rasteira como um urso para expulsar um monstro invasor. A selvageria está no sangue do druida.
			
			\begin{multicols}{2}
			\begin{itemize}
				\item \textbf{Forma Selvagem de Combate} Quando você escolhe esse círculo, no \textbf{2\textsuperscript{o}} nível, você recebe a habilidade de usar sua Forma Selvagem no seu turno com uma ação bônus, ao invés de com uma ação.
				Além disso, enquanto você estiver transformando pela sua Forma Selvagem, você pode usar uma ação bônus para gastar uma espaço de magia e ganhar 1d8 pontos de vida por nível do espaço de magia gasto.
				
				\item \textbf{Formas De Circulo} Os ritos do seu círculo garantem a você a habilidade de se transformar em formas animais mais poderosas. A partir do \textbf{2\textsuperscript{o} nível}, você pode usar sua Forma Selvagem para se transformar em uma besta com nível de desafio até 1 (você ignora a coluna ND Max da tabela Formas de Besta, mas ainda deve acatar as limitações descritas lá).
				A partir do \textbf{6\textsuperscript{o} nível}, você pode se transformar em uma besta com nível de desafio tão alto quanto seu nível de druida dividido por 3, arredondado para baixo.
				
				\item \textbf{Ataque Primordial} A partir do \textbf{6\textsuperscript{o} nível}, seus ataques na forma de besta contam como mágicos com os propósitos de ultrapassar resistência e imunidade a ataques e danos não-mágicos.
				
				\item \textbf{Forma Selvagem de Elemental} No \textbf{10\textsuperscript{o} nível}, você pode gastar dois usos da sua Forma Selvagem, ao mesmo tempo, para se transformar em um elemental da água, elemental do ar, elemental do fogo ou elemental da terra.
				
				\item \textbf{Mil Formas} No \textbf{14\textsuperscript{o} nível}, você aprende a usar magia para alterar sua forma física de formas mais sutis. Você pode conjurar a magia alterar-se a vontade.
			\end{itemize}
			
			
			\vfill\null
		\end{multicols}}
	
	\section{Observações}
	
		\begin{multicols}{2}
			\subsection{Plantas e Florestas Sagradas}
			O druida tem certas plantas como sagradas, em particular o amieiro, freixo, bétula, elder, avelã, azevinho, zimbro, visco, carvalho, sorva, salgueiro e teixo. Druidas, muitas vezes, usam essas plantas como parte de seu foco de conjuração, incorporando lascas de carvalho ou teixo ou ramos de visco branco. 
			
			Da mesma forma, um druida usa tais madeiras para fazer outros objetos, com armas e escudos. O teixo está associado a morte e renascimento, então, empunhaduras de cimitarras ou foices seriam feitas com esse material. O freixo está associado com a vida e o carvalho com a força. Essas madeiras fazem excelentes cabos ou armas inteiras, como clavas ou bordões, assim como escudos. O amieiro é associado ao ar e seria usado para armas de arremesso, como dardos e azagaias.
			
			Os druidas de regiões que não possuem as plantas descritas aqui, tem que escolher outras plantas para usos similares. Por exemplo, um druida de uma região desértica valorizaria a árvore da iúca e as plantas de cacto.
			
			\subsection{O Druida e Os Deuses}
			Alguns druidas veneram as próprias forças da natureza, mas, a maioria dos druidas são devotados de uma das muitas divindades da natureza adoradas no multiverso (a lista de deuses encontrada no apêndice B inclui muitos desses deuses). A adoração desses deuses é, muitas vezes, considerada uma tradição mais antiga que as crenças de clérigos e pessoas urbanizadas. De fato, no mundo de Greyhawk, a doutrina druídica é chamada de Crença Antiga, e ela reivindica muitos adoradores dentre os agricultores, silvicultores, pescadores e outros que vivem perto da natureza. Essa tradição inclui a adoração da Natureza como força primordial acima de personificação, mas também engloba a adoração de Beory, a Mãe Oerth, assim como a devoção a Obad-Hai, Ehlonna e Ulaa.
			
			Nos mundos de Greyhawk e dos Reinos Esquecidos, os círculos druídicos não estão normalmente conectados a fé de uma única divindade da natureza. Cada círculo dos Reinos Esquecidos, por exemplo, pode incluir druidas que reverenciam Silvanus, Mielikki, Eldath, Chauntea ou, até mesmo os ferozes Deuses da Fúria: Talos, Malar, Auril e Umberlee. Esses deuses da natureza são, muitas vezes, chamados de Primeiro Círculo, o primeiro entre os druidas, e muitos druidas consideram todos (até os mais violentos) como merecedores de veneração.
			
			Os druidas de Eberron possuem crenças animistas, completamente desconectados do Soberano Anfitrião, dos Seis Sombrios ou de qualquer outra religião do mundo. Eles acreditam que cada coisa viva e cada fenômeno natural – sol, lua, vento, fogo e o próprio mundo – tem um espirito. Suas magias, portanto, são um meio de se comunicar e de comandar esses espíritos. Diferentes seitas druídicas, no entanto, possuem diferentes filosofias sobre o relacionamento mais adequado com esses espíritos entre si e com as forças da civilização. O Ashbound, por exemplo, acredita que a magia arcana é uma abominação contra a natureza, as Crianças do Inverno veneram as forças da morte e os Guardiões do Portal preservam tradições antigas destinadas a proteger o mundo da incursão de aberrações.
			
			
			\vfill\null
		\end{multicols}
	%\section{Tabelas Úteis}
	%	\subsection{}
	\pagebreak
	\section{Magias de Druida}
		\begin{multicols}{4}
			{\setstretch{0.8} \paragraph{0 Nível (Truques)}
			\begin{itemize}
				\item Bordão Místico
				\item Chicote de Espinhos
				\item Consertar
				\item Criar Chamas
				\item Druidismo
				\item Orientação
				\item Rajada de Veneno
				\item Resistência
			\end{itemize}
		\textbf{1\textsuperscript{o} Nível}
			\begin{itemize}
				\item Amizade Animal
				\item Bom Fruto
				\item Constrição
				\item Criar ou Destruir Água
				\item Curar Ferimentos
				\item Detectar Veneno e Doença
				\item Enfeitiçar Pessoa
				\item Falar com Animais
				\item Fogo das Fadas
				\item Névoa Obscurecente
				\item Onda Trovejante
				\item Palavra Curativa
				\item Passos Longos
				\item Purificar Alimentos
				\item Salto
			\end{itemize}
		\columnbreak
		\textbf{2\textsuperscript{o} Nível}
			\begin{itemize}
				\item Aprimorar Habilidade
				\item Crescer Espinhos
				\item Encontrar Armadilhas
				\item Esfera Flamejante
				\item Esquentar Metal
				\item Imobilizar Pessoa
				\item Lâmina Flamejante
				\item Localizar Animais ou Plantas
				\item Localizar Objeto
				\item Lufada de Vento
				\item Mensageiro Animal
				\item Passos sem Pegadas
				\item Pele de Árvore
				\item Proteção contra Veneno
				\item Raio Lunar
				\item Restauração Menor
				\item Sentido Bestial
				\item Visão no Escuro
			\end{itemize}
		\textbf{3\textsuperscript{o} Nível}
			\begin{itemize}
				\item Ampliar Plantas
				\item Andar na Água
				\item Conjurar Animais
				\item Convocar Relâmpagos
				\item Dissipar Magia
				\item Falar com Plantas
				\item Forjar Morte
				\item Luz do Dia
				\item Mesclar-se às Rochas
				\item Muralha de Vento
				\item Nevasca
				\item Proteção contra Energia
			\end{itemize}
		\columnbreak
		\textbf{4\textsuperscript{o} Nível}
			\begin{itemize}
				\item Confusão
				\item Conjurar Elementais Menores
				\item Conjurar Seres da Floresta
				\item Controlar a Água
				\item Dominar Besta
				\item Inseto Gigante
				\item Localizar Criatura
				\item Malogro
				\item Metamorfose
				\item Moldar Rochas
				\item Movimentação Livre
				\item Muralha de Fogo
				\item Pele de Pedra
				\item Tempestade de Gelo
				\item Terreno Alucinógeno
				\item Vinha Esmagadora
			\end{itemize}
		\textbf{5\textsuperscript{o} Nível}
			\begin{itemize}
				\item Âncora Planar
				\item Caminhar em Árvores
				\item Conjurar Elemental
				\item Comunhão com a Natureza
				\item Cúpula Antivida
				\item Curar Ferimentos em Massa
				\item Despertar
				\item Missão
				\item Muralha de Pedra
				\item Praga
				\item Praga de Insetos
				\item Reencarnação
				\item Restauração Maior
				\item Vidência
			\end{itemize}
		\columnbreak
		\textbf{6\textsuperscript{o} Nível}
			\begin{itemize}
				\item Banquete de Heróis
				\item Caminhar no Vento
				\item Conjurar Fada
				\item Cura Completa
				\item Encontrar o Caminho
				\item Mover Terra
				\item Muralha de Espinhos
				\item Raio Solar
				\item Teletransporte por Árvores
			\end{itemize}
		\textbf{7\textsuperscript{o} Nível}
			\begin{itemize}
				\item Inverter a Gravidade
				\item Miragem
				\item Regeneração
				\item Tempestade de Fogo
				\item Viagem Planar
			\end{itemize}
		\textbf{8\textsuperscript{o} Nível}
			\begin{itemize}
				\item Antipatia/Simpatia
				\item Controlar o Clima
				\item Enfraquecer o Intelecto
				\item Explosão Solar
				\item Formas Animais
				\item Terremoto
				\item Tsunami
			\end{itemize}
		\textbf{9\textsuperscript{o} Nível}
			\begin{itemize}
				\item Alterar Forma
				\item Ressurreição Verdadeira
				\item Sexto Sentido
				\item Tempestade da Vingança
			\end{itemize}}
		\end{multicols}
\end{document}